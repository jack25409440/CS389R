\documentclass[12pt]{article}

\usepackage{graphicx}
\usepackage[margin=0.8in]{geometry}
\usepackage{amsmath}
\usepackage{cases}
\usepackage{amsfonts}
\usepackage{amssymb}
\usepackage{grffile}
\usepackage{setspace}
\usepackage{listings}


%\setlength\parindent{0pt}

\author{Xiaohui Chen \\Department of Computer Science\\ The University of Texas at Austin }
\title{CS 389R Project Proposal}
\date{\vspace{-5ex}}

\begin{document}
\maketitle

\begin{spacing}{1.0}

Electrons in atoms have different states. Each state can be represented by a set of quantum numbers in ket notation: $|n\ l\ m_l\ m_s \rangle$. Here $n$, $l$, $m_l$, $m_s$ represent principle quantum number, azimuthal quantum number, magnetic quantum number and spin projection quantum number respectively.

In quantum physics, the Clebsch-Gordan coefficients are sets of number that arise in angular momentum coupling. As for electrons, $\vec{l}$ is the angular momentum and $\vec{s}$ is the spin. We let $\vec{j}= \vec{l}+ \vec{s}$, where $\vec{j}$ is the total angular momentum. Therefore for any given state $|j\ m_j \rangle$, angular momentum $l$ and spin $s$,a Clebsch-Gordan coefficient $C_{m_l,m_s}$ indicates the probability of finding such electron in state $|l\ m_l\rangle |s\ m_s\rangle$. The Clebsch-Gordan coefficients could be calculated using the lowering operator method.

The raising and lowering operator $L_{\pm}$, $S_{\pm}$ obey the following properties:

\begin{equation}
L_{\pm} |l\ m_l \rangle = \hbar \sqrt{l(l+1)-m_l(m_l \pm 1)} |l\ (m_l \pm 1) \rangle
\label{l}
\end{equation}


\begin{equation}
S_{\pm} |s\ m_s \rangle = \hbar \sqrt{s(s+1)-m_s(m_s \pm 1)} |s\ (m_s\pm 1) \rangle
\label{s}
\end{equation}

Since $\vec{j}= \vec{l}+ \vec{s}$, $J_{\pm}$ obeys the following properties:

\begin{equation}
J_{\pm}= L_{\pm} + S_{\pm}
\label{jsum}
\end{equation}

\begin{equation}
J_{\pm} |j\ m_j \rangle = \hbar \sqrt{j(j+1)-m_j(m_j \pm 1)} |s\ (m_j \pm 1) \rangle
\label{j}
\end{equation}

Therefore a state can be written as 

\begin{equation}
|j\ m_j \rangle= \sum\limits_{m_l+m_s=m_j} C_{m_l m_s m_j}^{l s j} |l\ m_l \rangle |s\ m_s \rangle \label{sum}
\end{equation}

In equation \ref{sum}, the set of $C_{m_l m_s m_j}^{l s j}$ is a set of Clebsch-Gordan coefficients. The first task of this project is to use ACL2 to compare each $C_{m_l m_s m_j}^{l s j}$ with the Clebsch-Gordan coefficient table to verify its correctness. The required functions for using lowering operators will also be implemented in ACL2.

Using the raising and lowering operator described above, for each state $|j\ m_j \rangle$, we can get a complete set shown in equation \ref{sum}. See attachment for an example.

Since each $|C_{m_l m_s m_j}^{l s j}|^2$ means the probability of finding such an electron in state $|l\ m_l \rangle |s\ m_s \rangle$ and the states are exhaustive, $\sum\limits_{m_l+m_s=m_j} \left| C_{m_l m_s m_j}^{l s j} \right|^2=1$ must be true. Therefore the second task of this project is to use ACL2 to verify such theorem.

\clearpage

\end{spacing}

\section*{Attachment}

\begin{spacing}{2.0}

Example:

We have an electron with angular momentum $l=1$ and spin $s=\frac{1}{2}$

Therefore $m_l=-1,0,-1$ and $m_s= \pm \frac{1}{2}$

This means $j=\frac{3}{2}$ and $m_j= -\frac{3}{2}, - \frac{1}{2}, \frac{1}{2}, \frac{3}{2}$

We know that the only possibility of forming $|j\ m_j\rangle= |\frac{3}{2}\ \frac{3}{2}\rangle$ is $|\frac{3}{2}\ \frac{3}{2}\rangle = |1\ 1\rangle |\frac{1}{2}\ \frac{1}{2}\rangle $. This is an instance of equation $\ref{sum}$

Applying the lowering operator on the above equation

$J_{-}|\frac{3}{2}\ \frac{3}{2}\rangle =  \hbar \sqrt{\frac{3}{2} \left( \frac{3}{2}+1 \right) -  \frac{3}{2} \left( \frac{3}{2}-1 \right)} |\frac{3}{2}\ \frac{1}{2}\rangle = \sqrt{3} \hbar |\frac{3}{2}\ \frac{1}{2}\rangle$

$J_{-} |1\ 1\rangle |\frac{1}{2}\ \frac{1}{2}\rangle = (L_{-}+ S_{-}) |1\ 1\rangle |\frac{1}{2}\ \frac{1}{2}\rangle= L_{-}|1\ 1\rangle |\frac{1}{2}\ \frac{1}{2}\rangle + |1\ 1\rangle S_{-} |\frac{1}{2}\ \frac{1}{2}\rangle = \hbar\sqrt{1(1+1)-1(1-1)} |1\ 0\rangle |\frac{1}{2}\ \frac{1}{2}\rangle + \hbar \sqrt{\frac{1}{2} \left( \frac{1}{2}+1 \right) -  \frac{1}{2} \left( \frac{1}{2}-1 \right)} |1\ 1\rangle |\frac{1}{2}\ -\frac{1}{2}\rangle = \hbar \sqrt{2} |1\ 0\rangle |\frac{1}{2}\ \frac{1}{2}\rangle + \hbar |1\ 1\rangle |\frac{1}{2}\ -\frac{1}{2}\rangle$

$\sqrt{3} \hbar |\frac{3}{2}\ \frac{1}{2}\rangle = \hbar \sqrt{2} |1\ 0\rangle |\frac{1}{2}\ \frac{1}{2}\rangle + \hbar |1\ 1\rangle |\frac{1}{2}\ -\frac{1}{2}\rangle$

$\therefore |\frac{3}{2}\ \frac{1}{2}\rangle = \sqrt{\frac{2}{3}} |1\ 0\rangle |\frac{1}{2}\ \frac{1}{2}\rangle + \frac{1}{\sqrt{3}} |1\ 1\rangle |\frac{1}{2}\ -\frac{1}{2}\rangle$

The coefficients match the Clebsch-Gordan coefficient table

We can apply the lowering operator again on the above equation to get further results

The sum of probabilities is $\frac{2}{3}+ \frac{1}{3} = 1$. Therefore the lowering operator method is correct when $l=1$ and $s=\frac{1}{2}$

The task is to verify the lowering operator method is correct for all allowed $l$ and $s$

\end{spacing}
\end{document}