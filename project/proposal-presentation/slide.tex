\documentclass[landscape]{slides}



\usepackage{graphicx}
\usepackage[margin=0.5in]{geometry}
\usepackage{amsmath}
\usepackage{cases}
\usepackage{amsfonts}
\usepackage{amssymb}
\usepackage{grffile}
\usepackage{setspace}
\usepackage{listings}
%\usepackage[landscape]{geometry}

%\setlength\parindent{0pt}


\author{Xiaohui Chen \\Department of Computer Science\\ The University of Texas at Austin }

\title{CS 389R Project Proposal Presentation}

\date{\vspace{-5ex}}


\begin{document}

\maketitle

The raising and lowering operator $L_{\pm}$, $S_{\pm}$ obey the following properties:

\begin{equation}
L_{\pm} |l\ m_l \rangle = \hbar \sqrt{l(l+1)-m_l(m_l \pm 1)} |l\ (m_l \pm 1) \rangle
\label{l}
\end{equation}


\begin{equation}
S_{\pm} |s\ m_s \rangle = \hbar \sqrt{s(s+1)-m_s(m_s \pm 1)} |s\ (m_s\pm 1) \rangle
\label{s}
\end{equation}

Since $\vec{j}= \vec{l}+ \vec{s}$, $J_{\pm}$ obeys the following properties:

\begin{equation}
J_{\pm}= L_{\pm} + S_{\pm}
\label{jsum}
\end{equation}

\begin{equation}
J_{\pm} |j\ m_j \rangle = \hbar \sqrt{j(j+1)-m_j(m_j \pm 1)} |s\ (m_j \pm 1) \rangle
\label{j}
\end{equation}

Therefore a state can be written as 

\begin{equation}
|j\ m_j \rangle= \sum\limits_{m_l+m_s=m_j} C_{m_l m_s m_j}^{l s j} |l\ m_l \rangle |s\ m_s \rangle \label{sum}
\end{equation}

Expample:

We have an electron with angular momentum $l=1$ and spin $s=\frac{1}{2}$

Therefore $m_l=-1,0,-1$ and $m_s= \pm \frac{1}{2}$

This means $j=\frac{3}{2}$ and $m_j= -\frac{3}{2}, - \frac{1}{2}, \frac{1}{2}, \frac{3}{2}$

We know that the only possibility of forming $|j\ m_j\rangle= |\frac{3}{2}\ \frac{3}{2}\rangle$ is $|\frac{3}{2}\ \frac{3}{2}\rangle = |1\ 1\rangle |\frac{1}{2}\ \frac{1}{2}\rangle $. This is an instance of equation $\ref{sum}$

Applying the lowering operator on the above equation

$J_{-}|\frac{3}{2}\ \frac{3}{2}\rangle =  \hbar \sqrt{\frac{3}{2} \left( \frac{3}{2}+1 \right) -  \frac{3}{2} \left( \frac{3}{2}-1 \right)} |\frac{3}{2}\ \frac{1}{2}\rangle = \sqrt{3} \hbar |\frac{3}{2}\ \frac{1}{2}\rangle$

$J_{-} |1\ 1\rangle |\frac{1}{2}\ \frac{1}{2}\rangle = (L_{-}+ S_{-}) |1\ 1\rangle |\frac{1}{2}\ \frac{1}{2}\rangle= L_{-}|1\ 1\rangle |\frac{1}{2}\ \frac{1}{2}\rangle + |1\ 1\rangle S_{-} |\frac{1}{2}\ \frac{1}{2}\rangle = \hbar\sqrt{1(1+1)-1(1-1)} |1\ 0\rangle |\frac{1}{2}\ \frac{1}{2}\rangle + \hbar \sqrt{\frac{1}{2} \left( \frac{1}{2}+1 \right) -  \frac{1}{2} \left( \frac{1}{2}-1 \right)} |1\ 1\rangle |\frac{1}{2}\ -\frac{1}{2}\rangle = \hbar \sqrt{2} |1\ 0\rangle |\frac{1}{2}\ \frac{1}{2}\rangle + \hbar |1\ 1\rangle |\frac{1}{2}\ -\frac{1}{2}\rangle$

$\sqrt{3} \hbar |\frac{3}{2}\ \frac{1}{2}\rangle = \hbar \sqrt{2} |1\ 0\rangle |\frac{1}{2}\ \frac{1}{2}\rangle + \hbar |1\ 1\rangle |\frac{1}{2}\ -\frac{1}{2}\rangle$

$\therefore |\frac{3}{2}\ \frac{1}{2}\rangle = \sqrt{\frac{2}{3}} |1\ 0\rangle |\frac{1}{2}\ \frac{1}{2}\rangle + \frac{1}{\sqrt{3}} |1\ 1\rangle |\frac{1}{2}\ -\frac{1}{2}\rangle$

The coefficients match the Clebsch-Gordan coefficient table

We can apply the lowering operator again on the above equation to get further results

The sum of probabilities is $\frac{2}{3}+ \frac{1}{3} = 1$. Therefore the lowering operator method is correct when $l=1$ and $s=\frac{1}{2}$

The task is to verify the lowering operator method is correct for all allowed $l$ and $s$

\clearpage

From previous slides, we know that l is integer and s is half integer. Therefore all the real numbers here can by repredented as $\frac{x}{y}$, where $x$ and $y$ are integers

We can represent $\frac{x}{y}$ as a pair of nil-lists (k . a . b). Here a is a list of x nils and b is a list of y nils. The element k is either t or nil, which represents positive number and negative  number respetively

e.g. $\frac{2}{3}$ can be represented as ('t '(nil nil) . '(nil nil nil))

\clearpage

Now I only implemened the natural number without the sign

\begin{lstlisting}[language=Lisp,breaklines=true]
;append two nil-lists together
(defun nil-list-plus (x y)
	(if (consp x)
	    (cons nil (nil-list-plus (cdr x) y))
	    (mapnil y)))

;multiply two nil-lists
(defun nil-list-times (x y)
	(if (consp x)
	    (append (mapnil y) (nil-list-times (cdr x) y))
	    nil))

;add two natural numbers
(defun nat-plus (a b)
	(if (consp a)
	    (cons (nil-list-plus (nil-list-times (car a) (cdr b))
				 (nil-list-times (cdr a) (car b)))
	          (nil-list-times (cdr a) (cdr b)))
	    nil))

;multiply two natural numbers
(defun nat-times (a b)
	(if (consp a)
	    (cons (nil-list-times (car a) (car b))
	          (nil-list-times (cdr a) (cdr b)))
	    nil))

;determine if all elements in the list are nil
(defun true-nilp (x)
	(if (consp x)
	    (and (not (car x))
		 (true-nilp (cdr x)))
            (not x)))

;determine if a pair represent a natural number
(defun true-nil-natpair (x)
	(and (true-nilp (car x))
	     (true-nilp (cdr x))))

;determine if a pair represents a natural number
(defun equal-nat (x y)
	(if (consp x)
	    (equal (/ (len (car x)) (len (car y)))
		   (/ (len (cdr x)) (len (cdr y))))
	    nil))

(defthm equal-nilp
	(implies (true-nilp x)
		 (equal (mapnil x) x)))
\end{lstlisting}

\end{document}
